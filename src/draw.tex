\bluepage{Kreslení a uniformní proměnné}

\begin{frame}[fragile]
\frametitle{Uniformní proměnné}
  \begin{itemize}
  \item Uniformní proměnné jsou uloženy v konstantní paměti
  \item Narozdíl od vertex atributů se v průběhu kreslení nemění
  \item Každá invokace shaderu adresuje stejnou hodnotu
  \item Uniformní proměnné lze využít ve všech shader stage
  \item Uniformní proměnné jsou vhodné například pro uložení matic, barvy, světla
  \item Stejně jako objekty v OpenGL zastupuje integerová hodnota, tak i každá uniformní proměnná má svoje integerové jméno
  \item Toto jméno lze získat z Shader Programu pomocí specializovaných funkcí
  \end{itemize}
\end{frame}

\begin{frame}[fragile]
\frametitle{Způsob využívání uniformních proměnných}
  \begin{enumerate}
  \item Vytvoření Shader Programu
  \item Získání integerového jména pomocí jména proměnné v shaderu
  \item Aktivování Shader Programu
  \item Nahrání dat pomocí vhodné OpenGL funkce
  \end{enumerate}
{\scriptsize
\begin{minted}[bgcolor=bg]{packages/c_cpp.py:CppLexer -x}
GLuint program = 0;
GLint  colorUniform = -1;
void init(){
  //sestaveni programu
  program = createProgram(
      compileShader(GL_VERTEX_SHADER  ,loadFile("flag.vp")),
      compileShader(GL_FRAGMENT_SHADER,loadFile("flag.fp")));
  //ziskani integeroveho jmena z promenne "color" v shaderu
  colorUniform = glGetUniformLocation(program,"color");
}
\end{minted}
}
{\scriptsize
\begin{minted}[bgcolor=bg]{packages/graphics.py:GLShaderLexer -x}
#verion 450

layout(location=0)out vec4 fColor;
uniform vec3 color;//uniformni promenna
void main(){
  fColor = vec4(color,1);
}
\end{minted}
}
\end{frame}


\begin{frame}[fragile]
\frametitle{Vykreslení dat}
  \begin{enumerate}
  \item Aktivování programu
  \item Nastavení uniformních proměnných
  \item Aktivování Vertex Array Objectu
  \item Zavolání vykreslovacího příkazu
  \end{enumerate}
{\scriptsize
\begin{minted}[bgcolor=bg]{packages/c_cpp.py:CppLexer -x}
void draw(){
  //vymazani barevneho framebuffer
  glClear(GL_COLOR_BUFFER_BIT);

  //aktivovani program
  glUseProgram(program);

  //nastaveni uniformni promenne
  glProgramUniform3fv(program,colorUniform,1,0,0);

  //aktivovani vao
  glBindVertexArray(vao);

  glDrawArrays(//vykresleni
    GL_TRIANGLES,//typ primitiva
    0,//prvni Vertex
    3);//pocet Vertexu pro vykresleni 3 -> 1 trojuhelnik

  //deaktivovani vao
  glBindVertexArray(0);
\end{minted}
}
\end{frame}

