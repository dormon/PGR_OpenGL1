\begin{frame}\frametitle{GLSL}\scriptsize
\begin{itemize}
  \item GLSL - OpenGL Shading Language.
  \item It describes programs that are executed on GPU.
  \item C based.
  \item No recursion, classes, exceptions, std libs.
  \item Vector and matrix types, built-in variables and functions, synchronzation, variable qualifiers, swizzling.
  \item Every shader stage must have main function - entry point.
  \item Shader is execute in many instances -- threads in particular stage of pipeline
  \item Some parts of pipeline have special setttings.
\end{itemize}

\begin{itemize}
  \item GLSL - OpenGL Shading Language.
  \item Slouží pro popis programů, které běží na GPU.
  \item Je odvozený od C.
  \item Neobsahuje rekurzi, třídy, výjimky, std knihovny.
  \item Obsahuje vektorové a maticové typy, vestavěné funkce, vestavěné proměnné, synchronizační funkce, kvalifikátory, rozšířené adresování vektorů.
  \item Každý shader musí obsahovat main funkci.
  \item Každá main funkce je vykonávána v mnoha instancích v dané části pipeline.
  \item Některé části pipeline mají speciální nastavení.
\end{itemize}
\end{frame}

\begin{frame}[fragile]\frametitle{Types, swizzling / typy, swizzling}\scriptsize
\begin{minted}[bgcolor=bg]{packages/graphics.py:GLShaderLexer -x}
#version 450

void main(){
  //32 bit integer
  int a;
  //32 bit unsigned integer
  uint b;
  //32 bit float
  float c;
  //vector of 4 ints
  ivec4 d;
  //vector of 3 floats
  vec3 e = vec3(1,2,3);
  //matrix 3x3 of floats
  mat3 m;
  //zeroth element of e
  e[0] == e.x == e.r;
  //swizzling 
  vec2 f = e.xy; // (1,2)
  f = e.zz; // (3,3)
  //matrix vector multiplication
  e = m*e;
  //constructing ivec4 from vec3 and scalar
  d = ivec4(c,4);
  d = ivec4(c.xx,c.yy);
}
\end{minted}
\end{frame}

\begin{frame}[fragile]\frametitle{Built-in functions / vestavěné funkce}\tiny
\begin{minted}[bgcolor=bg]{packages/graphics.py:GLShaderLexer -x}
abs acos acosh asin asinh atan atanh ceil cos cosh degrees exp exp2 floor fract inversesqrt
log log2 max min mod modf pow radians round roundEven sign sin sinh sqrt tan tanh trunc
clamp cross distance dot floatBitsToInt floatBitsToUint fma frexp intBitsToFloat isinf
isnan ldexp length mix normalize smoothstep step

packDouble2x32 packSnorm4x8 packUnorm2x16 packSnorm2x16
packUnorm4x8 uintBitsToFloat unpackDouble2x32 unpackSnorm4x8 unpackUnorm2x16 
unpackSnorm2x16 unpackUnorm4x8 packHalf2x16 unpackHalf2x16

all any bitCount bitfieldExtract bitfieldInsert bitfieldReverse determinant equal
faceforward findLSB findMSB greaterThan greaterThanEqual imulExtended inverse lessThan
lessThanEqual matrixCompMult not notEqual outerProduct reflect refract transpose uaddCarry
umulExtended usubBorrow 

textureSize textureQueryLod texture textureProj textureLod 
textureOffset texelFetch texelFetchOffset textureProjOffset textureLodOffset textureProjLod 
textureProjLodOffset textureGrad textureGradOffset textureProjGrad textureProjGradOffset 
textureGather textureGatherOffset textureGatherOffsets textureQueryLevels

dFdx dFdy fwidth
interpolateAtCentroid interpolateAtOffset interpolateAtSample

noise1 noise2 noise3 noise4

EmitStreamVertex EndStreamPrimitive EmitVertex EndPrimitive

barrier memoryBarrier 
memoryBarrierAtomicCounter memoryBarrierBuffer memoryBarrierImage memoryBarrierShared 
groupMemoryBarrier
imageSize

atomicAdd atomicMin atomicMax atomicAnd atomicOr atomicXor atomicExchange atomicCompSwap

imageSize imageLoad imageStore imageAtomicAdd imageAtomicMin imageAtomicMax
imageAtomicAnd imageAtomicOr imageAtomicXor imageAtomicExchange imageAtomicCompSwap
\end{minted}
\end{frame}

\begin{frame}[fragile]\frametitle{Variable qualifiers / kvalifikátory proměnných}\tiny
\begin{minted}[bgcolor=bg]{packages/graphics.py:GLShaderLexer -x}
#version 450

//data comes from previous stage (or buffer in case of vertex shaders)
//read only
in vec4 a;

//data is sent to next stage
out vec4 b;

//a,b variables are different for every shader invocation

//data is constant, all threads see the same value, can be set from CPU
uniform mat4 m;

//opaque type d (texture), can be accessed through specialized functions
//read only
uniform sampler2D d;

//local variable stored in registers 
//every thread has its own
vec4 e = vec4(0,1,2,3);

void main(){
  b = e + m * texture(d,a.xy);
}
\end{minted}
\end{frame}

\begin{frame}[fragile]\frametitle{Vertex Shader built-ins / vestavěné proměnné VS}
Vertex Shader contains some important built-in variables.\\
Vertex Shader obsahuje několik důležitých vestavěných proměnných.
{\scriptsize
\begin{minted}[bgcolor=bg]{packages/graphics.py:GLShaderLexer -x}
#version 450

void main(){
  //output variable - contain position of vertex in clip-space
  vec4 gl_Position;

  //vertex/index number
  in int gl_VertexID;
  //instance number
  in int gl_InstanceID;
  //draw call number
  in int gl_DrawID;
  //point size
  out float gl_PointSize;
  //distance to custom clip plane
  out float gl_ClipDistance[];
  out float gl_CullDistance[];
}

\end{minted}
}
\end{frame}

\begin{frame}[fragile]
\frametitle{Fragment built-ins / vestavěné progměnné FS}
Output of fragment shader is specified manually.\\
Výstup fragment shaderu si specifikuje programátor pomocí vlastní výstupní proměnné.
{\scriptsize
\begin{minted}[bgcolor=bg]{packages/graphics.py:GLShaderLexer -x}
#version 450

//output, 0. color buffer of framebuffer
layout(location=0)out vec4 fColor;

void main(){
  //fragment coordinates (x,y in screen resolution)
  in vec4 gl_FragCoord;

  //Was this fragment rasterized from front facing side of a triangle?
  in bool gl_FrontFacing;

  //this variable can modify depth of fragment.
  //writes disable early depth test (can be re-enabled)
  out float gl_FragDepth;

  in float gl_ClipDistance[];
  in float gl_CullDistance[];
  in int gl_PrimitiveID;
}

\end{minted}
}
\end{frame}

